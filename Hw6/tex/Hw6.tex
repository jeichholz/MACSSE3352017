\documentclass{article}

\usepackage[margin=1in]{geometry}
\usepackage{mathtools}
\DeclarePairedDelimiter{\ceil}{\lceil}{\rceil}

\usepackage{listings}
\renewcommand{\tt}[1]{\texttt{#1}}
%\newcommand{\linux}[1]{\begin{lstlisting}[language=bash] #1 \end{lstlisting}}
\lstnewenvironment{linux}{\lstset{language=bash}}{}
\newcommand{\M}{\mathcal{M}}

\newcommand{\N}{\mathcal{N}}

\begin{document}
\begin{center}
\Large{MA/CSSE Homework 6}\\
Due 5/3
\end{center}
\vspace{.25in}
\noindent{\textbf{Directions}}\\
  \begin{itemize}
   \item \textbf{Each problem must be self-contained in a file}, named
     properly, and \textbf{must compile using the
      command} \texttt{mpicc filename.c}\\

  \item Turn in each \tt{.c} file to the dropbox. \textbf{Do not
      create a \tt{.zip} file}

  \item On this homework you may not use any of the standard default
    MPI collective operations. 
  \end{itemize}


\section*{Problems}

\noindent{\textbf{Level:}Easy}\\
\begin{enumerate}

\item \textit{Goldbach's Conjecture} is one of the most famous
  unsolved problems in mathematics.  First stated in 1742 in a letter
  from Christian Goldbach to Leonhard Euler, Goldbach's conjecture
  claims that every even integer greater than two  may be
  written as the sum of two prime numbers.  For example, $6$ may be
  written as $3+3$, $8$ as $5+3$, $10$ as $5+5$, and so on. No one has
  been able to prove that \textit{every} even integer greater than two
  is indeed expressible as a sum of two primes, however, no one has
  ever found a counterexample either. 

  Two prime numbers, $p$ and $q$ that sum to a given integer $n$ form a \textit{Goldbach partition}
  of $n$.  So, $3$ and $5$ form a Goldbach partition of $8$, and $11$
  and $13$ form a Goldbach partition of $24$.  Goldbach partitions are
  not unique, for example $3$ and $7$ form a Goldbach partition of
  $10$, as do $5$ and $5$. 

  Write a program in a file \tt{goldbach.c} that calculates Goldbach partitions for lists of
  numbers and prints the partitions.  Use a worker pool (dynamic work
  allocation) strategy to do so.  The standard code is supplied in
  \tt{objs/goldbach\_standard.o} and may be compiled with\\ 
  \tt{mpicc objs/golbach\_standard.o -o goldbach\_standard}\\
  You need to have a working copy of \tt{goldbach\_standard} in order
  for the testing code to run. 

  The file \tt{goldbach\_start.c} contains some starting code to get
  you going.  In particular, reading command line options and
  determining which numbers need to be checked is taken care of for
  you.  Once you call \tt{parse\_options}, the list of numbers that
  need to be checked is stored in the \tt{to\_partition} member of the
  \tt{options} struct. Run the standard code with \tt{-h} to see a
  full list of options for specifying which numbers to partition.
  Just by using \tt{parse\_options} you will automatically support
  most of the options provided.  

  Here are the requirements that must be met for this problem:
  \begin{itemize}
  \item Partitions must be formed using the \tt{goldbach\_partition}
    function that is supplied. 

  \item Every single number in the \tt{to\_partition} list must be
    passed to \tt{goldbach\_partition}, even if it is a negative or
      odd number.

  \item Partitions must be displayed using the format \tt{n: a/b}
    where \tt{n} is the integer being partitioned, \tt{a} is the
    smallest number in the partition, and \tt{b} is the largest number
    in the partition.  For example, the line for $n=8$ should read
    \tt{8:3/5}. Partitions \textbf{may be printed by the workers} --
    there is no need to send data back to the master processor. 

   \item \tt{goldbach\_partition} will return 2 if $n$ is odd or $\le
      2$.  Do not print result line if the result of
      \tt{goldbach\_partition} is 2.  However, if the result is 3,
      print out a result line and notify me immediately!

  \item Follow a worker-pool strategy for balancing the work.  The
    root processor should never call \tt{goldbach\_partition}.  The
    worker processes should all spend a similar amount of time running
    (within 10\%).  

  \item Respect the \tt{--chunksize} option.  The \tt{chunksize} is
    the number of integers to assign to a processor as part of a given
    assignment.  

  \item Respect the \tt{--print\_results} option. 
  
  \item In order to check to see if the load balancing is good, your
    code is run using 2,3,5,9,17,33 processors.  Your code should be
    nearly perfectly parallel when the number of integers to check is
    large (average efficiency of $\ge 60\%$ ). Notice that you'll need to run
    the tests on \tt{grendel} to get good efficiency with many
    processes. 

    In order to facilitate your testing, you can have the testing
    script not do a full efficiency test for you.  The testing script
    takes an option \tt{--max-procs=n} that tells the script not to test
    efficiency using more than $n$ processes.  For example,\\
\tt{./test\_goldbach.py --max-procs=4} tells the testing script not to
test efficiency using more than 4 processes. Setting
\tt{--max-procs=0} skips these tests altogther. 

  \item We might want to partition some big numbers, so make sure you
    can support partitioning \tt{long long int} types. 


  \end{itemize}

\end{enumerate}

\newpage

\noindent{\textbf{Level:} Difficult}

Complete the problem above.  In addition:

\begin{enumerate}
\item Many numbers have many different Goldbach partitions.  For
  example, 500 can be partitioned as:
\begin{align*}
&	13+487\\
&	37+463\\
&	43+457\\
&	61+439\\
&	67+433\\
&	79+421\\
&	103+397\\
&	127+373\\
&	151+349\\
&	163+337\\
&	193+307\\
&	223+277\\
&	229+271.
\end{align*}
For an even integer greater than 2, $n$, let's call the smallest
number appearing in any partition of $n$ the \textit{Goldbach minimum}
(that is not a standard name)
of $n$, and denote it as $\M(n)$. We can see from above that
$\M(500)=13$. The function \tt{goldbach\_partition} has the special
property that it returns $\M(n)$ in the variable \tt{a}. 

It is interesting that $\M(n)$ is usually a pretty small number. 
Indeed, from our first program we can tell that $\M(4)=2$, $\M(6)=3$,
$\dots$ , $\M(98)=19$, $\M(100)=3$.

Define $\N(p)$ to be the smallest even integer $n$ such that $\M(n)\ge
p$. For instance, $\N(2)=4$, $\N(3)=6$, $\N(5)=18$,$\N(7)=30$ and $\N(19)=98$.

Write a program using a worker-pool approach that prints out $\N(p)$
and the corresponding Goldbach partition of $\N(p)$ for a given value
of $p$. 

For example, when $p=2$ your program should print\\
\tt{N(2)=4: 2/2}\\
and when $p=7$ your program should print\\
\tt{N(7)=30: 7/23}.

In addition to submitting \tt{goldbach\_plus.c}, also turn in a file
\tt{big\_number.txt} that includes the output of your program when run
with \tt{--p=1900}. 

I've included a file \tt{goldbach\_plus\_helpers.c} to that reads the
command line options for you. You need not respect \tt{--chunksize}
though it might be useful for you to do so. 

I've included \tt{goldbach\_plus\_start.c} to start from; leave
\tt{goldbach\_partition} there and use it in your code. 

I've included the standard code in \tt{objs}, you can compile as you
did in the first problem.  Again, you a working copy of the standard
in order for the testing code to work. 

The important option to respect is \tt{--p}, which is the value of $p$
for which we are calculating $\N(p)$. 

Here are the grading criteria for this problem:
\begin{itemize}
\item Your program must always return the correct result, your format
  must match mine exactly. 
\item Every calculation of $\M(n)$ must be done using
  \tt{goldbach\_partition}, which is provided. 
  \item Follow a worker-pool strategy for balancing the work.  The
    root processor should never call \tt{goldbach\_partition}.  The
    worker processes should all spend a similar amount of time running
    (within 10\%).  
  \item In order to check to see if the load balancing is good, your
    code is run using 2,3,5,9,17,33 processors.  Your code should be
    nearly perfectly parallel when the number of integers to check is
    large (average efficiency of $\ge 60\%$ ). Notice that you'll need to run
    the tests on \tt{grendel} to get good efficiency with 3,5,9
    processes. As in the last problem, you can use a \tt{--max-procs}
    option to avoid this test, or to reduce the number of processors
    it uses. 
  \item There is no ambiguity about what list you need to partition
    here: it is just $1,2,3,4,\dots$. So your work assignments
    absolutely should not be lists of numbers to check, your work
    assignments should just be a range (top and bottom) of numbers to check.

\item You must respect the \tt{--p} option. 
\end{itemize}

\end{enumerate}



\end{document}
